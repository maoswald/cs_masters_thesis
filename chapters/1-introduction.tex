\chapter{Introduction}
\label{cap:introduction}

The number of used electronic devices connected over the internet grew throughout the last two decades rapidly.
% Through the establishment of smartphones tremendous masses of people are using personalized services based on an internet connection.
Through the establishment of smartphones, a large number of people are using services and transfer personal data via the internet. %+
Apart from connected end-user devices, control units of industrial facilities or plants that provide all different kinds of supply can be operated via an internet connection. %+

These are just some examples for connections, where authentication and identification is needed to restrict the access to the qualified entity.
Without restriction, every other entity has access and may be able to steal or cause damage otherwise. %+
Since data, transferred via the internet, passes multiple nodes and can be eavesdropped, encryption is needed to protect the information contained.
Authentication, identification, and encryption are just some security features needed to meet the requirements of information security.

The information security defines attributes a system has to meet to prevent it from threats, damage, and to minimize risks.
The key concepts are confidentiality, integrity and availability. %+
Data integrity means that data must not be changed without notice.
All changes have to be comprehensible.
Availability defines that data has to be available for a specified time frame. %+
Confidentiality requires that data is not accessible to unauthorized entities.
This applies all the time, even when data is transferred.
Hence, concepts, for example authentication, are required to ensure the information security goals \cite{Wikipedia2017InformationSecurity}.

To meet the requirements of information security, secret keys are used.
Secret keys have to be accessible by the devices to authenticate themselves and have to be protected. %+
The attackers aim is to get the secret key to impersonate the entity the key belongs to.
Different attacks to get to know the key are possible.
Intrusion into a system that stores the key via a connection or with physical access to the key storage device extracting the key are two examples for possible attacks \cite{Wikipedia2016Attackcomputing,Wikipedia2017Side-channelAttack}.

Approaches like noninvasive memory, a memory specially protected against physical attacks, or one-way hash functions have been used so far to prevent those attacks \cite{Pappu2001PhysicalFunctions}. %+
A key, stored as hash of an one-way hash function, makes it harder for attackers to get to know the key. 
Knowing the secret key, used as input of an one-way hash function, makes it is easy to calculate its hash that is stored to represent the key.
To verify a given key, its hash is calculated by the hash function and compared with the stored hash.
If the hashes match, the key will be verified to be correct.
When an attacker gets to know the hash, the key is still protected, since the one-way hash functions are hard to evaluate in the reverse direction with only the hash. %+

Though, due to the development of physical intrusion methods and the increasing performance of computer systems both techniques facing a number of challenges.
For that reason \pufs have been introduced as a replacement for secret key memory.
\pufs were also developed being an alternative to physical implementations of one-way functions used as authentication device such as smart cards \cite{Pappu2001PhysicalFunctions}.
It is suggested that \pufs offer a solution for security problems of the future with good prospects \cite{Tajik2014PhysicalPUFs}.

% old:
% Hence they are mostly stored in data, which makes them vulnerable to being red out and reused by unqualified entities.
% To protect keys one-way hash functions are.
% These functions are easy to evaluate in the one direction but infeasible to calculated in the reverse direction.
% The secret key is applied to the one-way hash function and only its result is stored.
% Hence the secret key can not be calculated from the stored value but conversely.
% However one-way function face 

However, \pufs are also not immune to attacks, e.g.\ physical intrusion or \acl{ML} attacks \cite{Tajik2014PhysicalPUFs,Ruhrmair2010ModelingFunctions,Becker2015ThePUFs,Helfmeier2014PhysicalFunctions}.
Since \pufs are not completely studied, new types of \pufs and suitable attacks are proposed. %+
This interplay between developing new \pufs that are claimed to be secure and developing of successful attacks leads to a huge variety of different \pufs \cite{Ruhrmair2014PUFOverview}.

Consequently, the \apuf and multiple modified versions based on the \apuf have been suggested. %+
One of these versions is based on the idea of \ac{MV} that has been suggested before, but has not been researched in detail \cite{Ruhrmair2013PUFData}.
This version of \apufs is not named yet, so we call it the \mpuf.
In combination with \xpufs it is then called \mxpufs.

The concept of \ac{MV} applied to \apufs is introduced to overcome the problems \xpufs face when growing large.
This thesis studies the impact of \ac{MV} on the possibility to build large \xpufs. %+
As there exist several successful attacks on \apufs, their impacts on \mpufs are unknown \cite{Ganji2016PACPUFs,Ruhrmair2014PUFOverview}.
%old: This thesis studies the changes due to the modification done to the \apuf and how existing attacks are affected.
Due to the modifications done to the \apuf, this thesis studies how existing attacks are affected by the use of \ac{MV}.

The thesis is structured as follows: 
Chap. \ref{cap:background} gives an introduction to \pufs and \ac{ML} attacks.
A more detailed description of important \ac{ML} attacks for this thesis is provided by Chap. \ref{cap:mla}. %+
In Chap. \ref{cap:arbiter} the \apuf specifications are outlined before in Chap. \ref{cap:attacks} applied attacks occurring in the current literature are shown.
% old: It is explained why only one of the attacks is relevant to the idea of \ac{MV}, which is introduced in Chap. \ref{cap:majorityarbiter}. %+
It is explained why only one of the attacks is relevant to the combination of \xpufs and the idea of \ac{MV}, which is introduced in Chap. \ref{cap:majorityarbiter}. %+
% alternative if Attacks chapter after MV chapter
% Chap. \ref{cap:arbiter} explains the \apuf specifications before Chap. shows the modifications made to create the \mpuf.
% After an overview to already applied attacks occurring in the current literature it is explained why only one of these attacks is relevant to the idea of \ac{MV}.

To prove the possibility of large \mxpufs and to research the impacts of \ac{MV} on the attacks in an empirical manner, simulations of the \pufs and the attacks are build and described in Chap. \ref{cap:simulationdesign}. %+
The results of the simulations are laid out in Chap. \ref{cap:stabilitysimulation} and Chap. \ref{cap:attacksimulations}.
Finally, a conclusion of the thesis summaries the results in Chap. \ref{cap:conclusion}.

% iot
% no security due to expenses (iot)
% bot nets
% ==============================
% - smart phones, cars, smart -everything
% - authentication
% - identification
% - encryption: for data, connections, signatures
% - attacks to steal secret keys
% - noninvasive memory
% - hash functions 
% - PUFs provide a alternative key storage developed to resist these attacks
% - (upon pufs Protocols are based )
% - gegenspiel zw attack and puf dev -> arbiter -> improvement by mv
% - Machine learning attacks on majority vote arbiter physical unclonable functions.
