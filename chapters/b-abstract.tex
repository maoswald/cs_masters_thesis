%%%%%%%%%%%%%%%%%%%%%%%%%
%% Abstract

\begin{abstract}

% the old abstract before the thesis started:
% This work shows possible attacks on current \apufs as well as further improvements of \apufs to resist these attacks.
% \apufs have been successfully attacked with different machine learning approaches which makes them unsafe and leads to the question how to make them robust to these attacks.
% Machine learning attacks in combination with a certain amount of \acp{CRP} made it so far possible to create models which can be used to calculate responses to new challenges.
% Beside an overview of the different current attacks this work simulates attacks to compare their success on \apufs and the improved \mpufs.
% \mpufs follow a new approach to make \apufs more robust and scaleable.

% The simulation shows that the robustness of \apufs ascends magnificent with the new majority based approach....

Since secret keys are used as basis for several security features, as e.g. encryption or authentication, there is a need to protect them from unauthorised access.
%\pufs use intrinsic imperfections that can not be red out easily and provide the possibility to derive secret keys for the purpose of e.g. identification of them.
Physical unclonable functions fulfill this need with the possibility to derive secret keys of their intrinsic imperfections that can be used but not be read out easily.
These imperfections are unique to ever instance and linked inseparable with the device.
As a result they can be used to securely identify devices.

Nevertheless, Arbiter \acsp{PUF} and \acs{XOR} Arbiter \acsp{PUF} can be successfully attacked by machine learning attacks.
Machine learning attacks in combination with a certain amount of \aclp{CRP} make it possible to create models, which can be used to predict responses for new challenges.
To counteract these attacks, the principle of \acl{MV} is applied to Arbiter \acsp{PUF}.
% Its impacts on the stability of \pufs and relevant attacks are verified by simulations.
Its impacts on the stability of Arbiter \acsp{PUF} and \acs{XOR} Arbiter \acsp{PUF} are verified by simulations.
Furthermore, relevant attacks are applied to Majority Arbiter \acsp{PUF} and Majority \acs{XOR} Arbiter \acsp{PUF} to study their success.
This work shows that \acl{MV} combined with Arbiter \acsp{PUF} enable large Majority \acs{XOR} Arbiter \acsp{PUF} that resist all by this time known non-invasive attacks.
Thus, large Majority \acs{XOR} Arbiter \acsp{PUF} can constitute as foundation to meet the goals of many security features.

% old with acronyms
% Since secret keys are used as basis for several security features, as e.g. encryption or authentication, there is a need to protect them.
% %\pufs use intrinsic imperfections that can not be red out easily and provide the possibility to derive secret keys for the purpose of e.g. identification of them.
% \pufs fulfill this need with the possibility to derive secret keys of their intrinsic imperfections that can be used but not be read out easily.
% These imperfections are unique to ever instance and linked inseparable with the device, which can be used to securely identify devices.

% However, \apufs and \acs{XOR} \apufs can be successfully attacked by machine learning attacks.
% Machine learning attacks in combination with a certain amount of \acf{CRP} make it possible to create models, which can be used to predict responses for new challenges.
% To counteract these attacks, the principle of \acl{MV} is applied to \apufs.
% % Its impacts on the stability of \pufs and relevant attacks are verified by simulations.
% Its impacts on the stability of \apufs and thus \xpufs are verified by simulations.
% Furthermore, relevant attacks are applied to \mpufs and \mxpufs to study their success.
% This work shows that \acl{MV} combined with \apufs enable large Majority \acs{XOR} \apufs that resist all by this time known non-invasive attacks.
% Thus, large Majority \acs{XOR} \apufs can constitute as foundation to meet the goals of many security features.

\end{abstract}

% http://www.ldeo.columbia.edu/~martins/sen_sem/thesis_org.html