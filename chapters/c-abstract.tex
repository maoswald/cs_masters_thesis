%%%%%%%%%%%%%%%%%%%%%%%%%
%% Abstract second language

\begin{abstract}

Geheime Schlüssel sind die elementare Basis für Schutzziele der Informationssicherheit, wie z. B. Verschlüsselung oder Authentifizierung, und müssen aus diesem Grund vor Fremdzugriffen geschützt werden.
Physical Unclonable Functions (\acsp{PUF}) sollen diesen Schutz durch intrinsische Abweichungen ihrer Hardware, welche zur Ableitung von geheimen Schlüsseln genutzt werden können, bieten.
Sind diese Abweichungen eindeutig für eine Instanz und an jedes Gerät fest gekoppelt, wie es das Konzept von \acsp{PUF} vorsieht, ist außerdem eine Identifizierung von einzelnen Geräten möglich.

% Arbiter \acsp{PUF} und \acs{XOR} Arbiter \acsp{PUF} können jedoch mittels Machine Learning Angriffen erfolgreich angegriffen werden.
% Machine Learning Angriffe nutzen Ein- und Ausgabepaare der zu angreifenden PUF, um ein Model dieser zu bilden.
% Das Model kann im Anschluss für das Vorhersagen von Ausgaben zu neuen Eingaben, der angegriffenen PUF, genutzt werden.
% Um diese Angriffe zu verhindern, wird das Prinzip Majority Vote in Kombination mit Arbiter \acsp{PUF} angewandt.
% Sein Einfluss auf die Stabilität von Arbiter \acsp{PUF} und \acs{XOR} Arbiter \acsp{PUF} wird mit Hilfe von Simulationen verifiziert.
% Zusätzlich wird der Erfolg von relevanten Angriffen auf Majority Arbiter \acsp{PUF} und Majority \acs{XOR} Arbiter \acsp{PUF} untersucht.
Jedoch können bestimmte \acsp{PUF} mittels Machine Learning Angriffen erfolgreich angegriffen werden.
Machine Learning Angriffe nutzen Ein- und Ausgabepaare der zu angreifenden \acs{PUF}, um ein Model dieser zu bilden.
Das Model kann im Anschluss für das Vorhersagen von Ausgaben zu neuen Eingaben, der angegriffenen \acs{PUF}, genutzt werden.
Arbiter \acsp{PUF} und kleine \acs{XOR} Arbiter \acsp{PUF} können somit erfolgreich angegriffen werden wohingegen große \acs{XOR} Arbiter \acsp{PUF} resistent wären, aber durch Instabilität unbrauchbar sind.

Um diese Angriffe zu verhindern, wird das Prinzip Majority Vote in Kombination mit Arbiter \acsp{PUF}, die für \acs{XOR} Arbiter \acsp{PUF} genutzt werden, angewandt.
Sein Einfluss auf die Stabilität von Arbiter \acsp{PUF}, \acs{XOR} Arbiter \acsp{PUF} und relevante Angriffe wird mit Hilfe von Simulationen verifiziert.
Es wird gezeigt, dass Realisierungen von großen Majority \acs{XOR} Arbiter \acsp{PUF} durch die Kombination von Majority Vote und Arbiter \acsp{PUF} möglich sind.
Darüber hinaus wird nachgewiesen, dass für große Majority \acs{XOR} Arbiter \acsp{PUF} die Komplexität der nicht invasiven Angriffe exponentiell steigt bei einem linearen Wachstum ihrer Größe.
Aus diesem Grund können große Majority \acs{XOR} Arbiter \acsp{PUF} als Fundament für Machine Learning resistente \acsp{PUF} dienen.
% Aus diesem Grund können große Majority \acs{XOR} Arbiter \acsp{PUF} das Fundament zum Erreichen vieler Sicherheitsziele bilden.
Da sie dennoch anfällig für invasive Angriffe sind, ist es ihnen nicht möglich einen allumfassenden Schutz für geheime Schlüssel zu bieten.

\end{abstract}
